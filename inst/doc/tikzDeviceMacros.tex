% !TEX root = tikzDevice.Rnw

% Font macros
\newcommand{\lang}{\textsf}
% Mbox it to prevent hyphenation of code
\newcommand{\code}[1]{\mbox{\ttfamily #1}}
\newcommand{\pkg}{\textbf}

% Other LaTeX macros
\renewcommand{\sectionautorefname}{Section}
\renewcommand{\subsectionautorefname}{Subsection}

% Text macros
\newcommand{\TikZ}{Ti\textit{k}Z}
\newcommand{\tikzDeviceVersion}{0.4-{\bfseries\color{red}Beta}}

% Source code formatting macros
\lstloadlanguages{R}
% I think by default, listings sets source code using \ttfamily. It looks horrid.
% In my opinion.
\lstset{ basicstyle = {\sffamily} }

\lstdefinestyle{sweaveChunk}{
	language = R,
	% The hacked version of the Sweave output function inserts these tags
	% around each output portion- we can grab them and apply formatting
	% to everything inside.
	moredelim=[is][\color{gray}]
		{swe@veSt@rtOutput}
		{swe@veEndOutput},
	% Normal TeX tildes are kind of ugly- let's substitute a math symbol instead.
	% Shit, why not replace <- as well? *Mad typographic scientist cackle*
	literate={~}{{$\sim$}}1 {<-}{$\leftarrow$}1,
	xleftmargin = 4ex,
	numbers = left,
	numberstyle = {\tiny},
	numberblanklines = false,
	numbersep = 5ex,
	showstringspaces = false,
	upquote = false,
	commentstyle = {\color{blue!80}\itshape}
}

\lstdefinestyle{latexSource}{
	language = [LaTeX]TeX,
	xleftmargin = 4ex,
	numbers = left,
	numberstyle = {\tiny},
	numberblanklines = false,
	numbersep = 5ex,
	showstringspaces = false,
	upquote = false,
	commentstyle = {\color{red!80}\itshape}
}


% Looking to make a nice code container- similar to that found
% at http://code.google.com/p/syntaxhighlighter/
% To do so, we need a customized shape that can have several
% different background colors.

% Building this I have finally started to learn what the hell
% \makeatletter does- it shifts the character @ into 'user land'
% so you can define macros with @ in their names. This way if a 
% user comes along and \newcommands the a macro with the same
% name the two don't clash since they won't be using the @ character
% anywhere. It's basically the LaTeX version of namespacing.

\makeatletter

% Shape 'code node'
%
% Inherits everything from rectangle except the behind background path.
%
% Adapted from Texample:
%		http://www.texample.net/tikz/examples/rectangle-node-with-diagonal-fill/
%
% Contributed by Mark Wibrow- the guy who designed the PGF math engine
% and made most of the stock node shapes.
%
\pgfdeclareshape{code node}
{
    % This bit from \pgflibarayshapes.code.tex
    \inheritsavedanchors[from=rectangle]
    \inheritanchorborder[from=rectangle]
    \inheritanchor[from=rectangle]{north}
    \inheritanchor[from=rectangle]{north west}
    \inheritanchor[from=rectangle]{north east}
    \inheritanchor[from=rectangle]{center}
    \inheritanchor[from=rectangle]{west}
    \inheritanchor[from=rectangle]{east}
    \inheritanchor[from=rectangle]{mid}
    \inheritanchor[from=rectangle]{mid west}
    \inheritanchor[from=rectangle]{mid east}
    \inheritanchor[from=rectangle]{base}
    \inheritanchor[from=rectangle]{base west}
    \inheritanchor[from=rectangle]{base east}
    \inheritanchor[from=rectangle]{south}
    \inheritanchor[from=rectangle]{south west}
    \inheritanchor[from=rectangle]{south east}

    \inheritbackgroundpath[from=rectangle]
    \inheritbeforebackgroundpath[from=rectangle]
    \inheritbehindforegroundpath[from=rectangle]
    \inheritforegroundpath[from=rectangle]
    \inheritbeforeforegroundpath[from=rectangle]
    
   % Now do the background filling.
    \behindbackgroundpath{%
        % \southwest and \northeast defined by rectangle, but
        % (somewhat annoyingly) not \southeast and \northwest
        % so use this workaround.
        \pgfextractx{\pgf@xa}{\southwest}%
        \pgfextracty{\pgf@ya}{\southwest}%
        \pgfextractx{\pgf@xb}{\northeast}%
        \pgfextracty{\pgf@yb}{\northeast}%
		
		% New code here. We want to draw a bar 4 ex wide to lie
		% under the line numbers. So we define two points that 
		% are 7 ex to the left of the southwest and northwest corners.
		\def\pgf@numberstripe@southwest{%
			\pgfpointadd{\southwest}%
			{\pgfpoint{\pgf@codestripe@width}{0}}%
		}
		\def\pgf@numberstripe@northwest{%
			\pgfpointadd{\pgfpoint{\pgf@xa}{\pgf@yb}}%
			{\pgfpoint{\pgf@codestripe@width}{0}}%
		}
		% Now we draw the code bar.
		\pgfpathmoveto{\southwest}%
		\pgfpathlineto{\pgf@numberstripe@southwest}%
		\pgfpathlineto{\pgf@numberstripe@northwest}%
		\pgfpathlineto{\pgfpoint{\pgf@xa}{\pgf@yb}}%
		\pgfpathclose		
		% Now fill it with a nice light grey.
		\color{\pgf@codestripe@color}%
		\pgfusepath{fill}%
		% Now for the rest of the node.
		\pgfpathmoveto{\pgf@numberstripe@southwest}%
		\pgfpathlineto{\pgfpoint{\pgf@xb}{\pgf@ya}}%
		\pgfpathlineto{\northeast}%
		\pgfpathlineto{\pgf@numberstripe@northwest}%
		\pgfpathclose
		%
		\color{\pgf@codebody@color}%
		\pgfusepath{fill}
    }
}

% Default widths.
\def\pgf@codestripe@width{4ex}

% Default color values
\def\pgf@codebody@color{white}
\def\pgf@codestripe@color{gray!30}

% Use these with PGF
\def\pgfsetcodestripecolor#1{\def\pgf@codestripe@color{#1}}%
\def\pgfsetcodestripewidth#1{\def\pgf@codestripe@width{#1}}%
\def\pgfsetcodebodycolor#1{\def\pgf@codebody@color{#1}}%

% Use these with TikZ
\tikzoption{codestripe color}{\pgfsetcodestripecolor{#1}}
\tikzoption{codestripe width}{\pgfsetcodestripewidth{#1}}
\tikzoption{codebody color}{\pgfsetcodebodycolor{#1}}

\makeatother




% TikZ Style definitions.
\tikzset{
	% Taken from one of the first examples in the PGF manuel.
	package warning/.style={
		rectangle split,
		rectangle split parts = 2,
		rounded corners,
		draw = red!50,
		thick,
		fill = red!10, 
		inner sep = 1ex,
		text width = \textwidth
	},
	code body/.style={
		% The code node is a homebrewed shape- several things can
		% possibly go pear shaped with it- for example, adding rounded
		% corners causes things to go whacky. Also- don't use the fill
		% command :P
		shape = code node,
		draw = blue!50,
		thick, 
		codebody color = blue!10,
		inner sep = 1ex,
		outer sep = 0pt,
		text width =\textwidth
	},
	code title body/.style={
		fill = white,
		inner sep = 1ex,
		outer sep = 0pt,
		above = -1ex of code node body.north west,
		anchor = south west
	},
	code title outline/.style={
		draw = blue!50,
		thick
	},
	code footer/.style={
		draw = blue!50,
		thick,
		rounded corners,
		fill = white,
		inner sep = 1ex,
		outer sep = 0pt,
	}
}



% TikZ Macros.

\newcommand{\tikzDocDisclaim}[2]{

	\begin{tikzpicture}
		
		\node[package warning]{
			\begin{center}
				\large\bfseries
				#1	
			\end{center}	
			\nodepart{second}		
				#2	
		};

	\end{tikzpicture}

}

% The following voodoo is discussed in Sweave.sty
\makeatletter 
\newwrite\lstvrb@out 
\newenvironment{tikzDocExample}[1][]{% 

  % Set up some pgfKeys to save input into this enviornment
  \pgfkeys{/code nodes/title/.store in=\@codeNodeTitle}
  \pgfkeys{/code nodes/footer/.store in=\@codeNodeFooter}
  
  % Now, store the arguments.
  % pgfKeys uses a unix-like path system.
  % Therefore we "cd" into the "directory" code nodes
  % so that users may specify options as:
  %  title = blah.
  %  footer = blah.
  %
  %  instead of:
  %    /code nodes/title = blah.
  %    /code nodes/footer = blah.
  \pgfkeys{ /code nodes/.cd, #1 }

  \begingroup 
  \@bsphack 
  \immediate\openout\lstvrb@out\jobname.lst 
  \let\do\@makeother\dospecials\catcode`\^^M\active 
  \def\verbatim@processline{% 
    \immediate\write\lstvrb@out{\the\verbatim@line}}% 
  \verbatim@start}{% 
  \immediate\closeout\lstvrb@out 
  \@esphack 
  \endgroup 
  
  % Normal, sane, LaTeX code resumes here.
  \begin{tikzpicture}
  	\node[code body,codebody color=white] (code node body) {
	    \lstinputlisting[style = latexSource]{\jobname.lst}%
	 };
	 
	 % Basically, \relax does jack squat if the macro \@codeNodeTitle
	 % is not defined. It's sole purpose in life is to allow us to get to the 
	 % \else statement without LaTeX freaking out.

	 \ifx \@codeNodeTitle \undefined
	   \relax
	 \else
	 	\node[code title body] (code node title) {
			\@codeNodeTitle
		};
		\draw[code title outline] (code node title.south west) --
			(code node title.south east) [rounded corners] --
			(code node title.north east) --
			(code node title.north west) [sharp corners] --
			cycle;
	 \fi
	 
	 % And the footer
	  \ifx \@codeNodeFooter \undefined
	   \relax
	 \else
	 	\node[code footer] at (code node body.south) {
			\@codeNodeFooter
		};
	 \fi
	 
  \end{tikzpicture}}
\makeatother